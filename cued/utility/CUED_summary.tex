\documentclass[11pt, a4paper]{scrartcl}
\usepackage{xcolor}
\usepackage[colorlinks=true,allcolors=blue,pdfborder={0 0 0},pdfstartview={FitV},breaklinks,linktocpage]{hyperref}
\usepackage{color,amsmath,amsfonts,amssymb}
\usepackage[utf8]{inputenc}
\usepackage{enumitem}%für mehrspaltige Aufzählungen
\usepackage[ngerman, english]{babel}
\usepackage{epsfig}
\usepackage{pstricks}
\usepackage{graphics}
\graphicspath{{Figures/}}
\usepackage{bbm}
%\usepackage{showlabels}
\usepackage[labelfont=bf]{caption}
%\usepackage{nonfloat}
\usepackage{booktabs}%für Tabellen
\usepackage{pgfplots} %für Plots
\pgfplotsset{
compat=1.5,
tick label style={font=\small},
every axis/.append style={line width=0.8pt, tick style={line width=0.7pt}}
} %für richtigen Abstand der labels
\usetikzlibrary{shapes,arrows,fit,calc,positioning,arrows}
\usepackage[
  textheight=25cm,
left=2.3cm,
right=2.3cm,
headheight=25pt,
  includehead,includefoot,
  heightrounded,
]{geometry}
\makeatletter
\def\input@path{{./Figures/}}
\usepackage{fancyhdr}
\pagestyle{fancy}
\fancyfoot{}
\fancyhead{}
% \renewcommand{\chaptermark}[1]
% {\markboth{{\chaptername\ \thechapter:\  #1}}{}}
% \renewcommand{\sectionmark}[1]
% {{\markright{ \thesection\ #1}}} 
%\renewcommand{\sectionmark}[1]{\markboth{#1}{}}

%\renewcommand\sectionmark[1]{\markright{#1}}
%\renewcommand{\thesection}{\arabic{section}.}
\fancyhead[L]{}
\fancyhead[R]{\thepage}
\fancyhead[C]{\nouppercase{\leftmark}}
%\fancyhead[C]{\nouppercase{\rightmark}}


%\usepackage{sectsty}
\usepackage{url}
%\usepackage{breakurl}
\usepackage{amsthm}
\usepackage{graphicx}
\usepackage{tikzpagenodes}

\definecolor{darkgreen}{rgb}{0.0,0.6,0.0}
\definecolor{shadedgreen}{rgb}{0.8,0.95,0.8}
\definecolor{shadedblue}{rgb}{0.85,0.85,1.0}
\newlength\figureheight 
\newlength\figurewidth


\usepackage[varg]{txfonts} %font/Schrift
%\usepackage{csquotes}
% \usepackage[minnames=3,maxnames=30,backend=biber,natbib=true,sorting=none,citestyle=numeric-comp,giveninits=true]{biblatex}
% \addbibresource{Literature.bib}
% %\setlength\bibitemsep{0.5em}
% \setlength{\biblabelsep}{0.4em}

% \DeclareFieldFormat{pages}{#1}
% \DeclareFieldFormat{journaltitle}{#1}
% \DeclareFieldFormat{eprint}{#1}
% \DeclareFieldFormat[article]{title}{\it#1\isdot}

% \DeclareBibliographyDriver{article}{
% \printnames{author}, \printfield{title},
% \href{http://dx.doi.org/\thefield{doi}}
% {\printfield{journaltitle} 
% \textbf{\printfield{volume}}, \printfield{pages} 
% (\printfield{year})}.}

% \DeclareBibliographyDriver{misc}{
% \printnames{author}, \printfield{title},
% \href{https://arxiv.org/abs/\thefield{eprint}}{
% arXiv preprint, arXiv:\printfield{eprint}  (\printfield{year})}.}

% \DeclareBibliographyDriver{inbook}{\hspace{-0.1cm}
% \printnames{author},
% \printfield{title},
% in \printfield{booktitle}.
% \printlist{publisher}
% (\printfield{year}).
% }


% \renewbibmacro*{name:andothers}{% Based on name:andothers from biblatex.def
%   \ifboolexpr{
%     test {\ifnumequal{\value{listcount}}{\value{liststop}}}
%     and
%     test \ifmorenames
%   }
%     {\ifnumgreater{\value{liststop}}{1}
%       {\finalandcomma}
%       {}%
%      \andothersdelim\bibstring[\emph]{andothers}}
%     {}}

\setenumerate[1]{label=[\arabic*],ref=\arabic*}
\newcommand{\paper}[4]{\item #1, \,\textit{#2}, \,\href{#3}{#4}.\\[-1.4em]}


\newcommand{\bE}{\mathbf{E}}
\newcommand{\bk}{\mathbf{k}}
\newcommand{\bA}{\mathbf{A}}
\newcommand{\bj}{\mathbf{j}}
\newcommand{\bd}{\mathbf{d}}
\newcommand{\sd}{\hspace{0.05em}}
\newcommand{\eqt}{\,{=}\,}
\newcommand{\pt}{\,{+}\,}
\newcommand{\coloneqqt}{\,{\coloneqq}\,}
\newcommand{\intbzdkpi}{\int\limits_\text{BZ}\frac{d\bk}{(2\pi)^2}}
\newcommand{\rhonnprime}{\rho_{nn'}(\bk,t)}
\newcommand{\rhonprimen}{\rho_{n'n}(\bk,t)}
\newcommand{\un}{\underline{n}}




\definecolor{darkgreen}{rgb}{0.0,0.5,0.0}


\begin{document}

\pagenumbering{Roman}

\begin{titlepage}\pdfbookmark[0]{Title}{Title}
  \sffamily
  \begin{center}
{
\includegraphics[width=9cm]{logo.pdf}
\\[5em]
\Huge \bfseries Output summary of the CUED program}
\\[3em]\large
The CUED program is developed and maintained by:
\\[3em]
Chair of Computational Condensed Matter Theory
  \\[0.5em]
Institute of Theoretical Physics
  \\[0.5em]
University of Regensburg
  \\[0.5em]
Universitätsstraße 31
  \\[0.5em]
D\,-\,93053 Regensburg
  \\[0.5em]
Germany
\\[3em]
Date of execution: \today
  \\[0.5em]
Run time: PH-RUN s
\\[3em]
  \end{center}{\large
  Responsible scientists:
  \\[1em]
    Jan Wilhelm
      \\[0.5em]
    Ferdinand Evers
  \\[3em]
  Contributors (in alphabetic order): 
  \\[1em]
  Jack Crewse
  \\[0.5em]
  Patrick Grössing
  \\[0.5em]
  Adrian Seith
  }
\end{titlepage}







\pagenumbering{arabic}
\pagestyle{plain}




\pdfbookmark[0]{Contents}{Contents}
\tableofcontents


\pagestyle{fancy}

\section{Electric-field pulse}
The following electric driving field is employed in the simulation:
\begin{align}
    \bE(t)  = E(t)\,\hat{e}_\phi
    \;,\hspace{2em}
    E(t) = E_0\,    \sin\Big(2\pi f_0\,(1+f_\text{chirp} t)\,t + \varphi\Big)\, e^{-t^2/(2\alpha)^2}\;.
    \label{e1}
\end{align}
\begin{figure}[b!]
\centering
\setlength\figureheight{7.5cm} 
\setlength\figurewidth{7.5cm}
\input{Efield.tikz}\hfill\input{Afield.tikz}
\caption{Left: Electric driving field~$E(t)$ from Eq.~\eqref{e1}, right: Gauge field~$A(t)$ from Eq.~\eqref{e2}.}
    \label{fig:Efield}
\end{figure}
The pulse is sketched in Fig.~\ref{fig:Efield}. 
%
The following parameters are used in the simulation:
\begin{itemize}
    \item Amplitude: $E_0 = PH-E0$\,MV/cm
    \item PH-EFIELD-DIRECTION
    \item Pulse frequency: $f_0 = PH-FREQ$\,THz
    \item Chirp: $f_\text{chirp} = PH-CHIRP$\,THz
    \item Carrier-envelope phase: $\phi = PH-CEP$
    \item $\alpha = PH-ALPHA$\,fs, full width at half maximum (FWHM) of the Gaussian envelope = PH-FWHM\,fs
\end{itemize}
The gauge field~$\bA(t)\eqt A(t)\,\hat{e}_\phi$ follows from~$\dot\bA(t)\eqt -\bE(t)$. We compute $A(t)$ for the sketch in Fig.~\ref{fig:Efield} as
\begin{align}
    \dot A(t) = -E(t) \hspace{2em}\Rightarrow\hspace{2em}
    A(t) = -\int\limits_{-\infty}^t E(t')\,dt'\;. \label{e2}
\end{align}


\section{Brillouin zone and \textit{k}-point grid}
You are using a PH-BZ as Brillouin zone (BZ) with a mesh size of PH-NK1\,$\times$\,PH-NK2. 
%
The BZ and a PH-SMALLNK1\,$\times$\,PH-SMALLNK2 $k$-point mesh is sketched in Fig.~\ref{fig:kp}.
%
Please note that in the params.py file, the BZ size (in case of a rectangular BZ) and the lattice parameter~$a$ (in case of a hexagonal BZ) are both given in atomic units, while the output here is in $1/\AA$ (note: 1 atomic length unit = $1 a_0$ = 0.529\,\AA, $a_0$:~Bohr radius, 1 atomic inverse length unit = 1/(0.529\,\AA) = 1.890\,\AA$^{-1}$). 
%
\begin{figure}[h!]
\centering
\setlength\figureheight{1.00000000000000000000\textwidth} 
\setlength\figurewidth{0.87\textwidth}
% This file was created with tikzplotlib v0.10.1.
\begin{tikzpicture}

\definecolor{darkgray176}{RGB}{176,176,176}
\definecolor{gray}{RGB}{128,128,128}
\definecolor{green}{RGB}{0,128,0}

\begin{axis}[
height=\figureheight,	 scale only axis=true,
tick align=outside,
tick pos=left,
width=\figurewidth,
x grid style={darkgray176},
xlabel={\(\displaystyle k_x \; (\si{\per\As})\)},
xmin=-0.519674684499713, xmax=0.519674684499713,
xtick style={color=black},
y grid style={darkgray176},
ylabel={\(\displaystyle k_y \; (\si{\per\As})\)},
ymin=-0.141729459409013, ymax=0.141729459409013,
ytick style={color=black}
]
\addplot [thick, black, mark=*, mark size=1, mark options={solid}, only marks]
table {%
0 0
};
\addplot [thick, black]
table {%
0.472431531363375 0.094486306272675
-0.472431531363375 0.0944863062726751
-0.472431531363375 -0.094486306272675
0.472431531363375 -0.094486306272675
0.472431531363375 0.094486306272675
};
\addplot [thick, black, mark=*, mark size=1, mark options={solid}, only marks]
table {%
0.472431531363375 2.08166817117217e-17
0 0.0944863062726751
};
\addplot [thick, gray, mark=*, mark size=1, mark options={solid}, only marks]
table {%
-0.45353427010884 0.0472431531363375
-0.41573974759977 0.0472431531363375
-0.3779452250907 0.0472431531363375
-0.34015070258163 0.0472431531363375
-0.30235618007256 0.0472431531363375
-0.26456165756349 0.0472431531363375
-0.22676713505442 0.0472431531363375
-0.18897261254535 0.0472431531363375
-0.15117809003628 0.0472431531363375
-0.11338356752721 0.0472431531363375
-0.07558904501814 0.0472431531363375
-0.03779452250907 0.0472431531363375
0 0.0472431531363375
0.03779452250907 0.0472431531363375
0.0755890450181401 0.0472431531363375
0.11338356752721 0.0472431531363375
0.15117809003628 0.0472431531363375
0.18897261254535 0.0472431531363375
0.22676713505442 0.0472431531363375
0.26456165756349 0.0472431531363375
0.30235618007256 0.0472431531363375
0.34015070258163 0.0472431531363375
0.3779452250907 0.0472431531363375
0.41573974759977 0.0472431531363375
0.45353427010884 0.0472431531363375
-0.45353427010884 0.0472431531363375
};
\addplot [thick, gray, mark=*, mark size=1, mark options={solid}, only marks]
table {%
-0.45353427010884 -0.0472431531363375
-0.41573974759977 -0.0472431531363375
-0.3779452250907 -0.0472431531363375
-0.34015070258163 -0.0472431531363375
-0.30235618007256 -0.0472431531363375
-0.26456165756349 -0.0472431531363375
-0.22676713505442 -0.0472431531363375
-0.18897261254535 -0.0472431531363375
-0.15117809003628 -0.0472431531363375
-0.11338356752721 -0.0472431531363375
-0.07558904501814 -0.0472431531363375
-0.03779452250907 -0.0472431531363375
0 -0.0472431531363375
0.03779452250907 -0.0472431531363375
0.0755890450181401 -0.0472431531363375
0.11338356752721 -0.0472431531363375
0.15117809003628 -0.0472431531363375
0.18897261254535 -0.0472431531363375
0.22676713505442 -0.0472431531363375
0.26456165756349 -0.0472431531363375
0.30235618007256 -0.0472431531363375
0.34015070258163 -0.0472431531363375
0.3779452250907 -0.0472431531363375
0.41573974759977 -0.0472431531363375
0.45353427010884 -0.0472431531363375
-0.45353427010884 -0.0472431531363375
};
\addplot [thick, green]
table {%
0.383948684086951 0.118107882840844
-0.142340736398547 0.118107882840844
};
\addplot [thick, red]
table {%
-0.496053107931544 0.118107882840844
-0.142340736398547 0.118107882840844
};
\addplot [thick, black, mark=*, mark size=1, mark options={solid}, only marks]
table {%
-0.142340736398547 0.118107882840844
};
\draw (axis cs:0.01,0.01) node[
  scale=1,
  anchor=base west,
  text=black,
  rotate=0.0
]{$\Gamma$};
\draw (axis cs:0.472431531363375,2.08166817117217e-17) node[
  scale=1,
  anchor=base west,
  text=black,
  rotate=0.0
]{$\mr{X}$};
\draw (axis cs:0,0.0944863062726751) node[
  scale=1,
  anchor=base west,
  text=black,
  rotate=0.0
]{$\mr{Y}$};
\end{axis}

\end{tikzpicture}

\caption{Brillouin zone (BZ) and PH-SMALLNK1\,$\times$\,PH-SMALLNK2 $k$-point mesh.
%
The BZ is indicated by the black PH-BZ.
%
Gray points indicate $k$-points, colored lines indicate $k$-points that are coupled via  the term $\bE(t)\cdot \nabla_\bk$.
%
The green and red line at the top left corner sketch ${\color{darkgreen}\bA_\text{max}}\coloneqqt\hat{e}_\phi\max (-qA(t)/\hbar)$ and ${\color{red}\bA_\text{min}}\coloneqqt\hat{e}_\phi\min (-qA(t)/\hbar)$ indicating the extremal excursion of  electrons in the BZ.  
}
    \label{fig:kp}
\end{figure}


\section{Hamiltonian, bandstructure and dipoles}
{\color{red}
This still needs to be filled; Gauge; how to print the Hamiltonian properly? (e.g.~Dirac or Bi$_2$Te$_3$?) Printing of band structure for rectangle along $k_x, k_y=0$, $k_y,k_x=0$, for hexagon along K-$\Gamma$-M. Plotting of dipoles as in Patricks plot with 4 diagrams (dvv, dcc, Re(dvc), Im(dvc)), additionally along K-$\Gamma$-M (also absolute value of dvc).
}

\section{Time evolution of the density matrix}
In case you choose the length gauge (\texttt{gauge = 'length'}, that is also the default), we solve semiconductor Bloch equations in the length gauge, Eq.~(50) in Ref.~[\ref{Wilhelm2021}]:
\begin{align}
    \left[
    \frac{\partial}{\partial t}
    +q\bE(t)\frac{\partial}{\partial \bk}
    \right]\rhonnprime = 
    \big[i\sd(\epsilon_{n'}(\bk)-\epsilon_{n}(\bk))
   \sd {-}1/T_2\big]\,\rhonnprime
    -i\sd \bE(t){\sum_{\un}}
    \big(\rho_{n\un}(\bk;t) \mathbf{d}_{\un n'}(\bk)
    -  \mathbf{d}_{n\un}(\bk)\rho_{\un n'}(\bk;t)
    \big)
\end{align}
with a dephasing time $T_2\eqt PH-T2$\,fs. {\color{red}TODO: Plots of initial vv and cc density matrix elements, plot of time evolution at a k-point, snapshot of density matrix at 3 time points for whole BZ.}

\section{Time-dependent current}
\begin{figure}[b!]
\centering
\setlength\figureheight{7.5cm} 
\setlength\figurewidth{7.5cm}
% This file was created with tikzplotlib v0.10.1.
\begin{tikzpicture}

\definecolor{darkgray176}{RGB}{176,176,176}
\definecolor{steelblue31119180}{RGB}{31,119,180}

\begin{axis}[
height=\figureheight,
tick align=outside,
tick pos=left,
width=\figurewidth,
x grid style={darkgray176},
xlabel={\(\displaystyle t \; (\si{\fs})\)},
xmajorgrids,
xmin=-92.0000000000078, xmax=92.4999999999927,
xtick style={color=black},
y grid style={darkgray176},
ylabel={Current \(\displaystyle j_{\parallel}(t)\) parallel to \(\displaystyle \bE\) in atomic units},
ymajorgrids,
ymin=-0.000255271091689042, ymax=0.00021728367372135,
ytick style={color=black}
]
\addplot [thick, steelblue31119180]
table {%
-92.0000000000078 -3.87846276457791e-08
-91.5000000000078 -4.18720129152121e-08
-91.0000000000078 -4.4895212877762e-08
-90.5000000000078 -4.78038619619107e-08
-90.0000000000078 -5.0541661773705e-08
-89.5000000000078 -5.30463192839502e-08
-89.0000000000078 -5.52495283542132e-08
-88.5000000000078 -5.70770300858007e-08
-88.0000000000078 -5.84488491567906e-08
-87.5000000000078 -5.92795437795268e-08
-87.0000000000078 -5.94785564587237e-08
-86.5000000000078 -5.8950775987837e-08
-86.0000000000078 -5.75971862549808e-08
-85.5000000000078 -5.53156251170419e-08
-85.0000000000078 -5.20017401338317e-08
-84.5000000000077 -4.7550092823509e-08
-84.0000000000077 -4.18553729771045e-08
-83.5000000000077 -3.48138079680434e-08
-83.0000000000077 -2.6324754134954e-08
-82.5000000000077 -1.62923944244034e-08
-82.0000000000077 -4.62760160803137e-09
-81.5000000000077 8.75000993669914e-09
-81.0000000000077 2.39099388672796e-08
-80.5000000000077 4.09085435090209e-08
-80.0000000000077 5.97866315847595e-08
-79.5000000000077 8.05669863843459e-08
-79.0000000000077 1.03251866145668e-07
-78.5000000000077 1.27820432329141e-07
-78.0000000000077 1.54226157806152e-07
-77.5000000000077 1.82394324334947e-07
-77.0000000000077 2.12219558107908e-07
-76.5000000000077 2.4356344529011e-07
-76.0000000000077 2.76252350822695e-07
-75.5000000000077 3.10075431345658e-07
-75.0000000000077 3.44782847048076e-07
-74.5000000000077 3.80084305636955e-07
-74.0000000000077 4.15647974293246e-07
-73.5000000000077 4.51099710961846e-07
-73.0000000000077 4.86022941764912e-07
-72.5000000000077 5.19958745430177e-07
-72.0000000000077 5.52406526311002e-07
-71.5000000000077 5.8282586725543e-07
-71.0000000000077 6.10638524593772e-07
-70.5000000000077 6.35231377173776e-07
-70.0000000000077 6.55960181545422e-07
-69.5000000000077 6.72154229760936e-07
-69.0000000000077 6.83121972575149e-07
-68.5000000000077 6.88157450625167e-07
-68.0000000000077 6.8654750646935e-07
-67.5000000000077 6.77579830948479e-07
-67.0000000000077 6.60551418338702e-07
-66.5000000000077 6.3477724181564e-07
-66.0000000000077 5.99599159881664e-07
-65.5000000000077 5.5439469878961e-07
-65.0000000000077 4.98585567276268e-07
-64.5000000000077 4.31646283266912e-07
-64.0000000000077 3.53113069813029e-07
-63.5000000000077 2.62593278188234e-07
-63.0000000000077 1.59776057007653e-07
-62.5000000000077 4.44448968741383e-08
-62.0000000000077 -8.3508009799581e-08
-61.5000000000077 -2.24065266762445e-07
-61.0000000000077 -3.77065813773984e-07
-60.5000000000077 -5.421858810907e-07
-60.0000000000077 -7.18920616707147e-07
-59.5000000000077 -9.06568180019774e-07
-59.0000000000077 -1.10421945656052e-06
-58.5000000000077 -1.31075661068268e-06
-58.0000000000077 -1.52486369990005e-06
-57.5000000000077 -1.74505363310663e-06
-57.0000000000077 -1.96971424814863e-06
-56.5000000000077 -2.1971733290668e-06
-56.0000000000077 -2.42577806023281e-06
-55.5000000000077 -2.65397560221111e-06
-55.0000000000076 -2.88037199604503e-06
-54.5000000000076 -3.10374176226989e-06
-54.0000000000076 -3.32297118676304e-06
-53.5000000000076 -3.5369572156667e-06
-53.0000000000076 -3.74455391851668e-06
-52.5000000000076 -3.94472730822768e-06
-52.0000000000076 -4.13706453593825e-06
-51.5000000000076 -4.3225902756113e-06
-51.0000000000076 -4.50447058898066e-06
-50.5000000000076 -4.68785808191387e-06
-50.0000000000076 -4.87828196416477e-06
-49.5000000000076 -5.07889599841618e-06
-49.0000000000076 -5.28817831982663e-06
-48.5000000000076 -5.50008402095936e-06
-48.0000000000076 -5.70704545394888e-06
-47.5000000000076 -5.90341398713181e-06
-47.0000000000076 -6.08590140201102e-06
-46.5000000000076 -6.25018222995221e-06
-46.0000000000076 -6.38686723046644e-06
-45.5000000000076 -6.48116900643502e-06
-45.0000000000076 -6.51713790507648e-06
-44.5000000000076 -6.48279540431496e-06
-44.0000000000076 -6.37175651478298e-06
-43.5000000000076 -6.18050756222597e-06
-43.0000000000076 -5.90416966619944e-06
-42.5000000000076 -5.53410727568693e-06
-42.0000000000076 -5.0588593164862e-06
-41.5000000000076 -4.46784961429754e-06
-41.0000000000076 -3.75622091128343e-06
-40.5000000000076 -2.92869415604049e-06
-40.0000000000076 -2.00039989907276e-06
-39.5000000000076 -9.93532190787292e-07
-39.0000000000076 6.93830749709272e-08
-38.5000000000076 1.17269251175589e-06
-38.0000000000076 2.31164514910459e-06
-37.5000000000076 3.49146043117822e-06
-37.0000000000076 4.72358736182493e-06
-36.5000000000076 6.02400033751823e-06
-36.0000000000076 7.41384493933503e-06
-35.5000000000076 8.91296388056398e-06
-35.0000000000076 1.0527957201516e-05
-34.5000000000076 1.22766705209323e-05
-34.0000000000076 1.42447481626885e-05
-33.5000000000076 1.65419526938193e-05
-33.0000000000076 1.9233589718662e-05
-32.5000000000076 2.24417281890751e-05
-32.0000000000076 2.62191412629949e-05
-31.5000000000076 3.04585226847193e-05
-31.0000000000076 3.50480835354937e-05
-30.5000000000076 3.96615282925204e-05
-30.0000000000076 4.4097700040904e-05
-29.5000000000076 4.81919922028961e-05
-29.0000000000076 5.20994354430136e-05
-28.5000000000076 5.60681703943831e-05
-28.0000000000076 6.03698940317993e-05
-27.5000000000076 6.49103425871205e-05
-27.0000000000076 6.93907049622452e-05
-26.5000000000075 7.34737429230567e-05
-26.0000000000075 7.72431766421116e-05
-25.5000000000075 8.09100986700615e-05
-25.0000000000075 8.4430964734719e-05
-24.5000000000075 8.74148302180632e-05
-24.0000000000075 8.97237389210937e-05
-23.5000000000075 9.17439307747654e-05
-23.0000000000075 9.33803315165664e-05
-22.5000000000075 9.42072015403103e-05
-22.0000000000075 9.44017823452787e-05
-21.5000000000076 9.4174897027335e-05
-21.0000000000076 9.32095092951245e-05
-20.5000000000076 9.16183289084939e-05
-20.0000000000076 8.95087307265969e-05
-19.5000000000076 8.6668598466481e-05
-19.0000000000076 8.35282543003404e-05
-18.5000000000076 7.99774668650973e-05
-18.0000000000076 7.56160612734393e-05
-17.5000000000076 7.12675290242971e-05
-17.0000000000076 6.68551220733768e-05
-16.5000000000076 6.12749527623792e-05
-16.0000000000076 5.51030982870394e-05
-15.5000000000076 4.89955264389107e-05
-15.0000000000076 4.17780600897613e-05
-14.5000000000076 3.19499467236461e-05
-14.0000000000076 1.95829302176746e-05
-13.5000000000076 5.49209862407509e-06
-13.0000000000076 -9.56166989633632e-06
-12.5000000000076 -2.48029002966641e-05
-12.0000000000076 -3.95395170128008e-05
-11.5000000000076 -5.38177767840425e-05
-11.0000000000076 -6.87086828478617e-05
-10.5000000000076 -8.50132208928781e-05
-10.0000000000076 -0.000101771979500635
-9.50000000000757 -0.000117682348769945
-9.00000000000757 -0.000133301336328592
-8.50000000000757 -0.000148527318118171
-8.00000000000757 -0.000161054524828362
-7.50000000000757 -0.000171731023089538
-7.00000000000757 -0.000182846388523717
-6.50000000000757 -0.000191862781329469
-6.00000000000757 -0.000200301444700357
-5.50000000000756 -0.000209445668824446
-5.00000000000756 -0.000216612278081924
-4.50000000000756 -0.000223887074067084
-4.00000000000756 -0.000228780671418532
-3.50000000000756 -0.000232684629844337
-3.00000000000756 -0.000233791329624934
-2.50000000000756 -0.000233772886340404
-2.00000000000756 -0.000233099294757625
-1.50000000000756 -0.000232287989176758
-1.00000000000756 -0.000231673761928027
-0.500000000007563 -0.000231604453370751
-7.5634022520587e-12 -0.000231815275623148
0.499999999992437 -0.000232233431464374
0.999999999992437 -0.000232411281265277
1.49999999999244 -0.000232666129090968
1.99999999999244 -0.000232557639590293
2.49999999999244 -0.000230919651112991
2.99999999999244 -0.00022847210616156
3.49999999999244 -0.00022276225218962
3.99999999999244 -0.000214818272243885
4.49999999999244 -0.000203876812583278
4.99999999999244 -0.000191353595355133
5.49999999999244 -0.000176903652215112
5.99999999999244 -0.000161359475715142
6.49999999999244 -0.000144233494617745
6.99999999999244 -0.000126152776320974
7.49999999999244 -0.000107444220635058
7.99999999999244 -8.88069132903596e-05
8.49999999999244 -6.95976300716354e-05
8.99999999999244 -4.94767030290212e-05
9.49999999999244 -2.91222937381546e-05
9.99999999999244 -9.50887441568562e-06
10.4999999999924 8.98657579971313e-06
10.9999999999924 2.68507889515123e-05
11.4999999999924 4.36365780769702e-05
11.9999999999924 5.96542649276977e-05
12.4999999999924 7.69109216236452e-05
12.9999999999924 9.26598875613843e-05
13.4999999999924 0.000106203496733052
13.9999999999924 0.000121056917267438
14.4999999999924 0.000133497183537632
14.9999999999924 0.000145353264428055
15.4999999999924 0.000155883498162269
15.9999999999924 0.000165111233684037
16.4999999999924 0.000173728218860617
16.9999999999924 0.000180824960030341
17.4999999999924 0.000186834359439921
17.9999999999924 0.000190982317514522
18.4999999999924 0.00019396473296626
18.9999999999924 0.000195344845981941
19.4999999999924 0.000195803911657241
19.9999999999924 0.000195364401553974
20.4999999999924 0.000194364324164953
20.9999999999924 0.000192670080482968
21.4999999999924 0.000190093846710105
21.9999999999924 0.000186939845330636
22.4999999999924 0.000182840631829171
22.9999999999924 0.000178255089050993
23.4999999999924 0.000172882145913132
23.9999999999924 0.000166936444202804
24.4999999999924 0.000160459658866686
24.9999999999924 0.000153024193890091
25.4999999999924 0.000144828925121187
25.9999999999924 0.000135508324221911
26.4999999999924 0.000126066380015777
26.9999999999924 0.000116138695936051
27.4999999999924 0.000105837329883176
27.9999999999924 9.60582250307558e-05
28.4999999999924 8.69157796653092e-05
28.9999999999924 7.76796160516385e-05
29.4999999999924 6.78650065461589e-05
29.9999999999924 5.78940099547115e-05
30.4999999999924 4.8144468327577e-05
30.9999999999924 3.91547502795126e-05
31.4999999999924 3.06805563461752e-05
31.9999999999924 2.22288723910197e-05
32.4999999999924 1.41315167893819e-05
32.9999999999924 7.00543878954873e-06
33.4999999999924 4.24188019707244e-07
33.9999999999925 -6.12384254479232e-06
34.4999999999925 -1.21198156634984e-05
34.9999999999925 -1.70489210646083e-05
35.4999999999925 -2.1468543874962e-05
35.9999999999925 -2.57310798278845e-05
36.4999999999925 -2.91360027772413e-05
36.9999999999925 -3.18010010930727e-05
37.4999999999925 -3.42921535029131e-05
37.9999999999925 -3.60353111170529e-05
38.4999999999925 -3.71659893678208e-05
38.9999999999925 -3.81450505507335e-05
39.4999999999925 -3.84336591729368e-05
39.9999999999925 -3.83774738213472e-05
40.4999999999925 -3.81488948571285e-05
40.9999999999925 -3.72063093698529e-05
41.4999999999925 -3.61187009722759e-05
41.9999999999925 -3.49475397294779e-05
42.4999999999925 -3.32510658023394e-05
42.9999999999925 -3.15273268338501e-05
43.4999999999925 -2.97732773897714e-05
43.9999999999925 -2.76629203185022e-05
44.4999999999925 -2.54486260148142e-05
44.9999999999925 -2.31340644829453e-05
45.4999999999925 -2.066071045684e-05
45.9999999999925 -1.82257818107117e-05
46.4999999999925 -1.57288383930008e-05
46.9999999999925 -1.30350840679324e-05
47.4999999999925 -1.03310693476531e-05
47.9999999999925 -7.6762954177674e-06
48.4999999999925 -4.92128732287737e-06
48.9999999999925 -2.11428358957221e-06
49.4999999999925 5.72703076262095e-07
49.9999999999925 3.14441756307686e-06
50.4999999999925 5.65558593030394e-06
50.9999999999925 8.05021872010279e-06
51.4999999999925 1.03018675919079e-05
51.9999999999925 1.24432940895146e-05
52.4999999999925 1.44355139651659e-05
52.9999999999925 1.62205675485293e-05
53.4999999999925 1.78258670854625e-05
53.9999999999925 1.92979639222063e-05
54.4999999999925 2.06357102223896e-05
54.9999999999925 2.18163141643371e-05
55.4999999999925 2.28331980941176e-05
55.9999999999925 2.37266017030414e-05
56.4999999999925 2.45388749089198e-05
56.9999999999925 2.52354228636906e-05
57.4999999999925 2.57813663809589e-05
57.9999999999925 2.62276086108098e-05
58.4999999999925 2.65915038277095e-05
58.9999999999925 2.68056866106192e-05
59.4999999999925 2.68729831993543e-05
59.9999999999925 2.68766741180759e-05
60.4999999999925 2.6826095421674e-05
60.9999999999925 2.66797442792375e-05
61.4999999999925 2.64717317729508e-05
61.9999999999925 2.62630816615019e-05
62.4999999999926 2.60321224517001e-05
62.9999999999926 2.57218800219287e-05
63.4999999999926 2.53390653422044e-05
63.9999999999926 2.4925232773414e-05
64.4999999999926 2.44838540115954e-05
64.9999999999926 2.40099751980122e-05
65.4999999999926 2.35383388598396e-05
65.9999999999926 2.30984235971282e-05
66.4999999999926 2.26752164670583e-05
66.9999999999926 2.2251562548951e-05
67.4999999999926 2.18353891664031e-05
67.9999999999926 2.14318486495866e-05
68.4999999999926 2.10312318310523e-05
68.9999999999926 2.06269839762515e-05
69.4999999999926 2.02273456972188e-05
69.9999999999926 1.98481812349883e-05
70.4999999999926 1.94936670441244e-05
70.9999999999926 1.91544738498624e-05
71.4999999999926 1.88327967892441e-05
71.9999999999926 1.85437585327578e-05
72.4999999999926 1.82855184075665e-05
72.9999999999926 1.80430017657283e-05
73.4999999999926 1.78196703721306e-05
73.9999999999926 1.76280666394495e-05
74.4999999999926 1.74603728005208e-05
74.9999999999926 1.7303475957507e-05
75.4999999999926 1.71642583371584e-05
75.9999999999926 1.70505734856043e-05
76.4999999999926 1.69515071946668e-05
76.9999999999926 1.68595780546132e-05
77.4999999999926 1.67850166669184e-05
77.9999999999926 1.67322080196408e-05
78.4999999999926 1.66913725507814e-05
78.9999999999926 1.66609599048946e-05
79.4999999999926 1.66479012815295e-05
79.9999999999926 1.664664256389e-05
80.4999999999926 1.66462458927454e-05
80.9999999999926 1.6649142995647e-05
81.4999999999926 1.66599626063864e-05
81.9999999999926 1.66721912922674e-05
82.4999999999926 1.66821818861523e-05
82.9999999999926 1.66967697423707e-05
83.4999999999926 1.67186498560343e-05
83.9999999999926 1.67431085573611e-05
84.4999999999926 1.67696658005282e-05
84.9999999999926 1.68006561627575e-05
85.4999999999926 1.68325838060416e-05
85.9999999999926 1.68605077283444e-05
86.4999999999926 1.68849913297229e-05
86.9999999999926 1.6908452876393e-05
87.4999999999926 1.6930635642293e-05
87.9999999999926 1.69506222377498e-05
88.4999999999926 1.69693839975145e-05
88.9999999999926 1.69887800660701e-05
89.4999999999926 1.70089429667883e-05
89.9999999999926 1.70278751426621e-05
90.4999999999927 1.70446225192913e-05
90.9999999999927 1.70605392253954e-05
91.4999999999927 1.70756478577578e-05
91.9999999999927 1.70877017075838e-05
92.4999999999927 1.70965076512437e-05
};
\end{axis}

\end{tikzpicture}
\hfill% This file was created with tikzplotlib v0.10.1.
\begin{tikzpicture}

\definecolor{darkgray176}{RGB}{176,176,176}
\definecolor{steelblue31119180}{RGB}{31,119,180}

\begin{axis}[
height=\figureheight,
tick align=outside,
tick pos=left,
width=\figurewidth,
x grid style={darkgray176},
xlabel={\(\displaystyle t \; (\si{\fs})\)},
xmajorgrids,
xmin=-92.0000000000078, xmax=92.4999999999927,
xtick style={color=black},
y grid style={darkgray176},
ylabel={Current \(\displaystyle j_{\bot}(t)\) orthogonal to \(\displaystyle \bE\) in atomic units},
ymajorgrids,
ymin=-1.25583812290442e-14, ymax=1.37769335846789e-14,
ytick style={color=black}
]
\addplot [thick, steelblue31119180]
table {%
-92.0000000000078 -5.68130214758217e-21
-91.5000000000078 -5.89327270036023e-21
-91.0000000000078 -6.38553914883065e-21
-90.5000000000078 -5.57729614985033e-21
-90.0000000000078 -6.69762150881175e-21
-89.5000000000078 -6.83189062357314e-21
-89.0000000000078 -7.08982978906713e-21
-88.5000000000078 -1.03232738080828e-20
-88.0000000000078 -9.39300211140267e-21
-87.5000000000078 -1.10178343358961e-20
-87.0000000000078 -9.33567276102836e-21
-86.5000000000078 -8.75697244729488e-21
-86.0000000000078 -8.6841950683766e-21
-85.5000000000078 -1.13333603189068e-20
-85.0000000000078 -1.16524479307207e-20
-84.5000000000077 -1.12399855685965e-20
-84.0000000000077 -1.00154718393441e-20
-83.5000000000077 -9.51512735928087e-21
-83.0000000000077 -1.07696789218133e-20
-82.5000000000077 -1.19167808635843e-20
-82.0000000000077 -1.07502830637525e-20
-81.5000000000077 -1.19902448215489e-20
-81.0000000000077 -1.44821834704538e-20
-80.5000000000077 -1.75668114811625e-20
-80.0000000000077 -2.53519312882797e-20
-79.5000000000077 -1.31162344576432e-21
-79.0000000000077 -9.72367574462454e-23
-78.5000000000077 3.36664035401705e-21
-78.0000000000077 2.44078850994812e-21
-77.5000000000077 -4.41298682072898e-22
-77.0000000000077 -3.11629646922385e-21
-76.5000000000077 -4.96225005784811e-21
-76.0000000000077 -1.01512428457695e-20
-75.5000000000077 -1.34416729256362e-20
-75.0000000000077 -1.76419918867432e-20
-74.5000000000077 -1.75008998927082e-20
-74.0000000000077 -9.51117952976408e-21
-73.5000000000077 -2.07525382564534e-20
-73.0000000000077 6.90286573249463e-21
-72.5000000000077 2.51466441534064e-20
-72.0000000000077 6.00729202437189e-20
-71.5000000000077 4.97495177020649e-20
-71.0000000000077 5.10567642064084e-20
-70.5000000000077 8.76431884309135e-20
-70.0000000000077 7.22940272696195e-20
-69.5000000000077 4.05246416346492e-20
-69.0000000000077 8.70087893920409e-20
-68.5000000000077 3.77288917577128e-20
-68.0000000000077 -9.00791708875368e-21
-67.5000000000077 -2.27394980167319e-20
-67.0000000000077 -5.57282480381066e-20
-66.5000000000077 -4.6149097182749e-20
-66.0000000000077 -3.81188686560674e-20
-65.5000000000077 -1.72611464897057e-19
-65.0000000000077 -1.00339408156922e-19
-64.5000000000077 -6.11000424971317e-20
-64.0000000000077 -7.05253996460959e-20
-63.5000000000077 -2.14146750888005e-19
-63.0000000000077 -6.90753447001014e-20
-62.5000000000077 -3.54824051178956e-20
-62.0000000000077 -3.34567222994003e-19
-61.5000000000077 -7.4721770777686e-20
-61.0000000000077 -1.07655594475349e-21
-60.5000000000077 -2.90977692496229e-19
-60.0000000000077 -6.4439562978816e-20
-59.5000000000077 -1.71809540570863e-20
-59.0000000000077 3.85802497752475e-20
-58.5000000000077 6.98443132320682e-20
-58.0000000000077 -2.64744880291419e-21
-57.5000000000077 -6.63839548382177e-20
-57.0000000000077 -1.1480700182264e-19
-56.5000000000077 -2.6073525866045e-19
-56.0000000000077 -4.2552521506093e-19
-55.5000000000077 -5.3074208076347e-19
-55.0000000000076 -6.82866026915983e-19
-54.5000000000076 -9.01846294290637e-19
-54.0000000000076 -1.02593584482059e-18
-53.5000000000076 -1.14343423650511e-18
-53.0000000000076 -1.28422138944349e-18
-52.5000000000076 -1.44012426666738e-18
-52.0000000000076 -1.29292171911946e-18
-51.5000000000076 -1.30909202882024e-18
-51.0000000000076 -1.3632274134707e-18
-50.5000000000076 -1.32561386699279e-18
-50.0000000000076 -9.74085395236544e-19
-49.5000000000076 -8.96749131450171e-19
-49.0000000000076 3.10751169032517e-19
-48.5000000000076 1.87610745376306e-18
-48.0000000000076 2.37879316837448e-18
-47.5000000000076 3.99687872386846e-19
-47.0000000000076 1.08270769300922e-19
-46.5000000000076 -1.66062050057648e-18
-46.0000000000076 -1.94043716409445e-18
-45.5000000000076 -1.70210086024372e-18
-45.0000000000076 -1.95098301824714e-18
-44.5000000000076 -2.57248535631217e-18
-44.0000000000076 -3.46817990234708e-18
-43.5000000000076 -3.67980004234433e-18
-43.0000000000076 -2.68848975199173e-18
-42.5000000000076 1.78646769346522e-18
-42.0000000000076 6.58483133293791e-18
-41.5000000000076 8.86625111463592e-18
-41.0000000000076 4.73719768538711e-18
-40.5000000000076 -1.1519588019452e-17
-40.0000000000076 -3.37727463955091e-17
-39.5000000000076 -4.45485001687249e-17
-39.0000000000076 -3.3258108712858e-17
-38.5000000000076 -6.34297974436962e-18
-38.0000000000076 3.01520031364198e-17
-37.5000000000076 6.33306664146609e-17
-37.0000000000076 7.30644897975912e-17
-36.5000000000076 5.7677385531877e-17
-36.0000000000076 2.52467748415332e-17
-35.5000000000076 -1.25516756125283e-17
-35.0000000000076 -4.17222112558048e-17
-34.5000000000076 -6.30684261747307e-17
-34.0000000000076 -7.38603502576474e-17
-33.5000000000076 -5.34070176569822e-17
-33.0000000000076 -3.78399308137288e-17
-32.5000000000076 -4.05762284378794e-17
-32.0000000000076 7.70409798700978e-17
-31.5000000000076 -6.6406540542528e-17
-31.0000000000076 1.17199592483531e-17
-30.5000000000076 -8.21929811521573e-17
-30.0000000000076 7.20831629158365e-16
-29.5000000000076 4.83391019853668e-16
-29.0000000000076 7.4900312088491e-16
-28.5000000000076 3.5888807184093e-16
-28.0000000000076 -3.88070559452246e-16
-27.5000000000076 6.15006091906979e-17
-27.0000000000076 1.25558095873562e-15
-26.5000000000075 1.43551010394711e-15
-26.0000000000075 7.74555022439928e-16
-25.5000000000075 -6.06068832041048e-16
-25.0000000000075 -1.39662543051532e-15
-24.5000000000075 -1.13259098926532e-15
-24.0000000000075 -1.78087136019486e-16
-23.5000000000075 -6.02966944806271e-16
-23.0000000000075 -9.32998747457675e-16
-22.5000000000075 1.48427694577691e-15
-22.0000000000075 1.40169868941971e-15
-21.5000000000076 -8.76880742226508e-17
-21.0000000000076 1.38029201160364e-15
-20.5000000000076 -9.61390999064483e-16
-20.0000000000076 -2.62492215538711e-15
-19.5000000000076 1.30091969490885e-15
-19.0000000000076 1.76433546088344e-16
-18.5000000000076 -1.01881669021058e-15
-18.0000000000076 2.61645453755944e-15
-17.5000000000076 1.00555984848371e-15
-17.0000000000076 -2.56748521532991e-15
-16.5000000000076 -4.28234796521227e-16
-16.0000000000076 1.5580023091014e-15
-15.5000000000076 -5.22601911707246e-16
-15.0000000000076 -2.18198221983537e-15
-14.5000000000076 -6.9220174430465e-16
-14.0000000000076 6.55535918586587e-16
-13.5000000000076 8.42447825361041e-16
-13.0000000000076 7.03445030945301e-16
-12.5000000000076 3.77772181343675e-16
-12.0000000000076 3.24598578591939e-16
-11.5000000000076 8.82083363608501e-17
-11.0000000000076 2.05748208405041e-16
-10.5000000000076 7.27618940891034e-16
-10.0000000000076 6.88661149537121e-16
-9.50000000000757 -4.51556601451423e-16
-9.00000000000757 -4.29241574064337e-16
-8.50000000000757 -1.12441303089838e-16
-8.00000000000757 -2.30200810123182e-15
-7.50000000000757 4.04547964425959e-15
-7.00000000000757 2.70439852450953e-15
-6.50000000000757 -1.60203617158748e-15
-6.00000000000757 -1.9317080090838e-15
-5.50000000000756 2.15032497178289e-15
-5.00000000000756 -2.19836344661921e-15
-4.50000000000756 1.30115029758633e-15
-4.00000000000756 1.60583267908244e-15
-3.50000000000756 1.13387336513029e-15
-3.00000000000756 -9.28563864257998e-16
-2.50000000000756 -2.23087279968723e-15
-2.00000000000756 2.49492973875942e-15
-1.50000000000756 -5.14747359534325e-16
-1.00000000000756 1.23042558129674e-15
-0.500000000007563 -4.901066197781e-16
-7.5634022520587e-12 -3.29115578842618e-15
0.499999999992437 7.03646948899478e-15
0.999999999992437 -5.38920425811726e-15
1.49999999999244 -2.74614042135976e-17
1.99999999999244 5.50407813823165e-15
2.49999999999244 -9.65052800849073e-15
2.99999999999244 1.25798738204187e-14
3.49999999999244 -1.1361321464784e-14
3.99999999999244 8.59106900248958e-15
4.49999999999244 -6.40836404011628e-15
4.99999999999244 5.76431325975891e-15
5.49999999999244 -4.34534749180654e-15
5.99999999999244 3.98207515686586e-15
6.49999999999244 -3.27640846597906e-15
6.99999999999244 4.15063446518687e-15
7.49999999999244 -3.66406281572075e-15
7.99999999999244 1.59548368087394e-15
8.49999999999244 2.8916563353002e-15
8.99999999999244 -1.25487790179211e-15
9.49999999999244 1.84918037282653e-15
9.99999999999244 2.80117571889791e-15
10.4999999999924 -4.19471894777288e-16
10.9999999999924 -2.4921793799964e-15
11.4999999999924 -2.09225809270431e-15
11.9999999999924 2.5470909395125e-16
12.4999999999924 1.80342542694274e-15
12.9999999999924 4.66791564194561e-15
13.4999999999924 2.90946898602103e-15
13.9999999999924 -1.06901776820515e-15
14.4999999999924 -6.06890002551071e-16
14.9999999999924 -2.12660664273656e-17
15.4999999999924 1.281700930301e-15
15.9999999999924 -4.95391920656649e-15
16.4999999999924 4.59354909068864e-15
16.9999999999924 -4.73630902141051e-15
17.4999999999924 2.71526216040069e-16
17.9999999999924 -1.76006649912244e-15
18.4999999999924 1.09260673477388e-15
18.9999999999924 -1.84549072998698e-15
19.4999999999924 1.24879786534461e-16
19.9999999999924 1.00088873815101e-15
20.4999999999924 -2.00249178215664e-15
20.9999999999924 2.64578326101501e-15
21.4999999999924 -6.44191391746069e-16
21.9999999999924 7.2477577861147e-16
22.4999999999924 8.33524626633981e-16
22.9999999999924 -9.68137533494429e-16
23.4999999999924 -2.94957697853453e-16
23.9999999999924 -1.58577587059792e-15
24.4999999999924 -7.57119210240818e-16
24.9999999999924 -3.14019540151518e-15
25.4999999999924 -2.48944027014448e-16
25.9999999999924 -1.42857233802853e-15
26.4999999999924 -3.95579207608959e-16
26.9999999999924 2.30253961228112e-16
27.4999999999924 6.24893884760539e-16
27.9999999999924 1.00248045907113e-15
28.4999999999924 2.74822146991255e-16
28.9999999999924 8.09122926238549e-16
29.4999999999924 -8.27121183992469e-16
29.9999999999924 -3.98998876582204e-16
30.4999999999924 8.59135584972253e-17
30.9999999999924 -5.06510344384133e-16
31.4999999999924 9.90764718164434e-16
31.9999999999924 9.22056369188847e-16
32.4999999999924 -1.33935159966902e-15
32.9999999999924 -9.57372325575366e-16
33.4999999999924 7.88919881909775e-16
33.9999999999925 1.39992698592205e-16
34.4999999999925 -8.83922560572729e-16
34.9999999999925 5.17899866869035e-17
35.4999999999925 8.002419109281e-16
35.9999999999925 9.97328457789067e-17
36.4999999999925 -4.14049919628919e-16
36.9999999999925 1.49616142035004e-16
37.4999999999925 4.58781214602957e-16
37.9999999999925 -1.3370455728943e-16
38.4999999999925 -2.61098475453892e-16
38.9999999999925 3.34851961056125e-16
39.4999999999925 -9.78711509924856e-17
39.9999999999925 -3.75522399124435e-16
40.4999999999925 1.23844886713611e-16
40.9999999999925 -4.15860994315407e-16
41.4999999999925 -2.39719919915564e-16
41.9999999999925 2.43915763754447e-16
42.4999999999925 -3.48058182683064e-16
42.9999999999925 -7.19030397269086e-17
43.4999999999925 2.17390831389608e-16
43.9999999999925 -4.04353358263995e-16
44.4999999999925 -1.5939707023315e-17
44.9999999999925 3.94983015320861e-16
45.4999999999925 -9.28372632769363e-17
45.9999999999925 -2.33583638912587e-17
46.4999999999925 2.6631234574698e-16
46.9999999999925 6.27014304502172e-17
47.4999999999925 -1.64700931815008e-16
47.9999999999925 -6.13740589408657e-17
48.4999999999925 9.09361973016785e-17
48.9999999999925 2.34483551800283e-17
49.4999999999925 -1.55718676304693e-16
49.9999999999925 -1.46353957817106e-16
50.4999999999925 3.76557298945286e-17
50.9999999999925 -1.34930688598916e-17
51.4999999999925 -2.19168159342807e-16
51.9999999999925 -6.4096295426146e-17
52.4999999999925 1.86135731908817e-16
52.9999999999925 1.24637934945893e-17
53.4999999999925 -1.14452045948284e-16
53.9999999999925 8.96369480700675e-17
54.4999999999925 4.4438823285537e-17
54.9999999999925 -2.19241277264932e-16
55.4999999999925 -1.24210476324237e-16
55.9999999999925 5.44897253499914e-17
56.4999999999925 -5.29654978964564e-17
56.9999999999925 -7.90348493618992e-17
57.4999999999925 9.2314188910963e-17
57.9999999999925 5.99735695093888e-17
58.4999999999925 -1.42411214477888e-16
58.9999999999925 -1.56708580481158e-16
59.4999999999925 5.81006258118715e-18
59.9999999999925 4.8156588402831e-17
60.4999999999925 -7.23811184484971e-17
60.9999999999925 -6.40175530484726e-17
61.4999999999925 7.23080005263718e-17
61.9999999999925 7.78424647857021e-18
62.4999999999926 -1.27112695387057e-16
62.9999999999926 -3.40335805215522e-17
63.4999999999926 2.66205481091566e-17
63.9999999999926 -8.21732955577389e-17
64.4999999999926 -9.15436385008733e-17
64.9999999999926 1.11139241630453e-17
65.4999999999926 2.13616821716833e-17
65.9999999999926 -3.05182958039898e-17
66.4999999999926 -6.29939021387184e-17
66.9999999999926 -5.8634949088941e-17
67.4999999999926 -1.12770333739402e-17
67.9999999999926 -1.23175576503387e-18
68.4999999999926 -5.02263880445317e-17
68.9999999999926 -4.04173375686456e-17
69.4999999999926 3.25093530680172e-18
69.9999999999926 -2.27340493254196e-17
70.4999999999926 -3.95624203253344e-17
70.9999999999926 -5.43322405946446e-18
71.4999999999926 -2.60862248320871e-17
71.9999999999926 -6.14134301297024e-17
72.4999999999926 -3.23518683126704e-17
72.9999999999926 -1.68564932776553e-17
73.4999999999926 -3.65870833403896e-17
73.9999999999926 -2.28352895252854e-17
74.4999999999926 -1.1940719128616e-17
74.9999999999926 -3.22281302906122e-17
75.4999999999926 -3.03776844152873e-17
75.9999999999926 -2.70986268307451e-17
76.4999999999926 -4.64636272828529e-17
76.9999999999926 -3.32236589226259e-17
77.4999999999926 -1.13614002071617e-17
77.9999999999926 -3.74476250392489e-17
78.4999999999926 -4.47875395295192e-17
78.9999999999926 -4.0721058168243e-18
79.4999999999926 -9.98340859787725e-18
79.9999999999926 -4.88596453463435e-17
80.4999999999926 -3.28018247565184e-17
80.9999999999926 -1.43029904588179e-17
81.4999999999926 -3.45229081542369e-17
81.9999999999926 -3.70932843397186e-17
82.4999999999926 -1.78407729985728e-17
82.9999999999926 -2.37070801352409e-17
83.4999999999926 -3.51134759867874e-17
83.9999999999926 -2.67892817755996e-17
84.4999999999926 -2.01355508621975e-17
84.9999999999926 -2.64630633538098e-17
85.4999999999926 -3.3285527933655e-17
85.9999999999926 -3.33305235780398e-17
86.4999999999926 -2.5287552144257e-17
86.9999999999926 -2.17778918822427e-17
87.4999999999926 -3.00964616378823e-17
87.9999999999926 -3.21887591017755e-17
88.4999999999926 -2.36958312241447e-17
88.9999999999926 -2.57600064102974e-17
89.4999999999926 -3.17894227578604e-17
89.9999999999926 -2.68230285088882e-17
90.4999999999927 -2.54787836328924e-17
90.9999999999927 -3.19412830576591e-17
91.4999999999927 -2.91403041947054e-17
91.9999999999927 -2.35214731021536e-17
92.4999999999927 -2.66149236536085e-17
};
\end{axis}

\end{tikzpicture}

\caption{Components of the time-dependent current~$\bj(t)$: left: parallel to the driving field, right: orthogonal to the driving field, see Eq.~\eqref{currentcomp}.}
    \label{fig:current}
\end{figure}
In case you choose the length gauge (\texttt{gauge = 'length'}, that is also the default), the current is computed from Eq.~(67) in Ref.~[\ref{Wilhelm2021}] as
\begin{align}
 \bj(t)  =
q\sum_{nn'} \intbzdkpi
\ \langle u_{n\bk} |  \frac{\partial h(\bk)}{\partial \bk} | u_{n'\bk}\rangle \ \rhonprimen\;.\label{current}
\end{align} 
The matrix element $ \langle u_{n\bk} |  (\partial_\bk h(\bk)) | u_{n'\bk}\rangle$ can be computed from Eq.~(68) in Ref.~[\ref{Wilhelm2021}] as
\begin{align}
   \langle u_{n\bk} |  \frac{\partial h(\bk)}{\partial \bk} | u_{n'\bk}\rangle  = \delta_{nn'}\partial_\bk\epsilon_n(\bk) 
   +\frac{i}{q}\, \bd_{nn'}(\bk)\,(\epsilon_n(\bk)-\epsilon_{n'}(\bk)) \;.
\end{align}
%
In our case, the current is a two-dimensional vector. 
%
For generating meaningful plots, we project the current onto the axis~$\hat{e}_\phi$ of the incoming E-field and its orthogonal direction $\hat{e}_{\phi+\pi/2}$:
\begin{align}
    j_{\parallel}(t) = \hat{e}_\phi\;\bj(t)\;,\hspace{2em}
    j_{\bot}(t) = \hat{e}_{\phi+\pi/2}\;\bj(t)\;, 
    \label{currentcomp}
\end{align}
and we recover 
\begin{align}
\bj(t)\eqt\hat{e}_\phi\, j_{\parallel}(t)\pt\hat{e}_{\phi+\pi/2}\,j_{\bot}(t)\;.
    \label{currentdecomp}
\end{align}

\section{Frequency-resolved emission spectrum}
\begin{figure}[b!]
\centering
\setlength\figureheight{9cm} 
\setlength\figurewidth{0.95\textwidth}
% This file was created with tikzplotlib v0.10.1.
\begin{tikzpicture}

\definecolor{darkgray176}{RGB}{176,176,176}
\definecolor{lightgray204}{RGB}{204,204,204}
\definecolor{steelblue31119180}{RGB}{31,119,180}

\begin{axis}[
height=\figureheight,
legend cell align={left},
legend style={fill opacity=1.0, draw opacity=1, text opacity=1, draw=lightgray204},
log basis y={10},
tick align=outside,
tick pos=left,
width=\figurewidth,
x grid style={darkgray176},
xlabel={Harmonic order = \(\displaystyle f/f_0\)},
xmajorgrids,
xmin=0, xmax=30, xtick={0,1,2,3,4,5,6,7,8,9,10,11,12,13,14,15,16,17,18,19,20,21,22,23,24,25,26,27,28,29,30}, xticklabels={,1,,,,5,,,,,10,,,,,15,,,,,20,,,,,25,,,,,30},
xtick style={color=black},
y grid style={darkgray176},
ylabel={Emission intensity in atomic units},
ymajorgrids,
ymin=3.65267619266265e-24, ymax=7.08330718935421e-16,
ymode=log,
ytick style={color=black}
]
\addplot [thick, steelblue31119180]
table {%
0 0
0.049968769500847 3.62898304411282e-20
0.0999375390016941 1.84208346024877e-19
0.149906308502541 5.67032566051983e-19
0.199875078003388 1.40811277011478e-18
0.249843847504235 3.04965904096872e-18
0.299812617005082 5.96545916976778e-18
0.349781386505929 1.07581737990646e-17
0.399750156006776 1.81325273218858e-17
0.449718925507623 2.88369516709886e-17
0.49968769500847 4.35688808487595e-17
0.549656464509317 6.28446972014188e-17
0.599625234010164 8.68454254718427e-17
0.649594003511011 1.15262418604681e-16
0.699562773011858 1.47179262132455e-16
0.749531542512705 1.81030973107998e-16
0.799500312013552 2.14674249824454e-16
0.849469081514399 2.45581786694588e-16
0.899437851015246 2.71143546328307e-16
0.949406620516093 2.89026967845761e-16
0.99937539001694 2.97526637339714e-16
1.04934415951779 2.95829967956666e-16
1.09931292901863 2.8414145570882e-16
1.14928169851948 2.63639933646358e-16
1.19925046802033 2.36281524894901e-16
1.24921923752118 2.04494457617009e-16
1.29918800702202 1.70831478173681e-16
1.34915677652287 1.37647266137901e-16
1.39912554602372 1.06853820762223e-16
1.44909431552456 7.97824754099074e-17
1.49906308502541 5.7154938809842e-17
1.54903185452626 3.91444386085198e-17
1.5990006240271 2.54956600161532e-17
1.64896939352795 1.5669409202204e-17
1.6989381630288 8.98283807894106e-18
1.74890693252965 4.72535446435787e-18
1.79887570203049 2.24066062191207e-18
1.84884447153134 9.74188176396407e-19
1.89881324103219 4.91184722197477e-19
1.94878201053303 4.7361589313372e-19
1.99875078003388 7.04402534871034e-19
2.04871954953473 1.04620405480596e-18
2.09868831903557 1.41989253804152e-18
2.14865708853642 1.7857462455327e-18
2.19862585803727 2.12865311967162e-18
2.24859462753812 2.44742093862191e-18
2.29856339703896 2.74762090955959e-18
2.34853216653981 3.03712839185649e-18
2.39850093604066 3.3235254386037e-18
2.4484697055415 3.61267100746787e-18
2.49843847504235 3.90793796133484e-18
2.5484072445432 4.20980801366921e-18
2.59837601404405 4.5156783513102e-18
2.64834478354489 4.81985324378003e-18
2.69831355304574 5.11376450083294e-18
2.74828232254659 5.38648486162008e-18
2.79825109204743 5.62557219990169e-18
2.84821986154828 5.81822063671014e-18
2.89818863104913 5.95261561958263e-18
2.94815740054997 6.01931753905381e-18
2.99812617005082 6.01245659281043e-18
3.04809493955167 5.93052786000757e-18
3.09806370905252 5.77663457794611e-18
3.14803247855336 5.55812841697746e-18
3.19800124805421 5.28571366267667e-18
3.24797001755506 4.97218675490758e-18
3.2979387870559 4.63104594066848e-18
3.34790755655675 4.27521223712458e-18
3.3978763260576 3.91605362116479e-18
3.44784509555845 3.56281633452271e-18
3.49781386505929 3.22246707133171e-18
3.54778263456014 2.8998650221811e-18
3.59775140406099 2.59813307155331e-18
3.64772017356183 2.31908993014053e-18
3.69768894306268 2.06363351463025e-18
3.74765771256353 1.83201504653464e-18
3.79762648206437 1.62399471357804e-18
3.84759525156522 1.43890821046337e-18
3.89756402106607 1.27569138458737e-18
3.94753279056692 1.13290820335356e-18
3.99750156006776 1.00881209431074e-18
4.04747032956861 9.01451322783379e-19
4.09743909906946 8.0881271058957e-19
4.1474078685703 7.28987974163298e-19
4.19737663807115 6.60342659510936e-19
4.247345407572 6.01666383785603e-19
4.29731417707284 5.52282541583673e-19
4.34728294657369 5.12095560994233e-19
4.39725171607454 4.81556159919487e-19
4.44722048557539 4.61532591962689e-19
4.49718925507623 4.53089986722864e-19
4.54715802457708 4.57199056751077e-19
4.59712679407793 4.74415037675767e-19
4.64709556357877 5.04581469072736e-19
4.69706433307962 5.46615827206389e-19
4.74703310258047 5.98422750236676e-19
4.79700187208131 6.56957650651001e-19
4.84697064158216 7.18434708967394e-19
4.89693941108301 7.78646231397656e-19
4.94690818058386 8.33341961430948e-19
4.9968769500847 8.78610965534924e-19
5.04684571958555 9.11214989568264e-19
5.0968144890864 9.28837097704982e-19
5.14678325858724 9.30227608488818e-19
5.19675202808809 9.15245785151465e-19
5.24672079758894 8.84807265469271e-19
5.29668956708978 8.407531801859e-19
5.34665833659063 7.85658728691249e-19
5.39662710609148 7.22599019707155e-19
5.44659587559233 6.54890288768181e-19
5.49656464509317 5.85826020975116e-19
5.54653341459402 5.18429468378323e-19
5.59650218409487 4.55244997438077e-19
5.64647095359571 3.98188897154714e-19
5.69643972309656 3.4847465157288e-19
5.74640849259741 3.0661837932852e-19
5.79637726209825 2.7251862490045e-19
5.8463460315991 2.45593344436826e-19
5.89631480109995 2.24948402956658e-19
5.9462835706008 2.09548298063485e-19
5.99625234010164 1.98362097049277e-19
6.04622110960249 1.90465207487817e-19
6.09618987910334 1.85088732298719e-19
6.14615864860418 1.81620062935055e-19
6.19612741810503 1.79568188089466e-19
6.24609618760588 1.78512741988695e-19
6.29606495710672 1.7805612861542e-19
6.34603372660757 1.77793633526245e-19
6.39600249610842 1.77308991318915e-19
6.44597126560927 1.76194747060131e-19
6.49594003511011 1.74090119985624e-19
6.54590880461096 1.7072534799691e-19
6.59587757411181 1.65960990628756e-19
6.64584634361265 1.59812793064201e-19
6.6958151131135 1.52456378036344e-19
6.74578388261435 1.44210174759723e-19
6.7957526521152 1.35498896501207e-19
6.84572142161604 1.26803179413469e-19
6.89569019111689 1.18603472912141e-19
6.94565896061774 1.11327610197813e-19
6.99562773011858 1.05311234370979e-19
7.04559649961943 1.00778040288903e-19
7.09556526912028 9.78426797883161e-20
7.14553403862113 9.65339166144301e-20
7.19550280812197 9.68305702965408e-20
7.24547157762282 9.86995062870415e-20
7.29544034712367 1.0212453892957e-19
7.34540911662451 1.07117812990216e-19
7.39537788612536 1.1371016178251e-19
7.44534665562621 1.2192252161942e-19
7.49531542512705 1.31725042153638e-19
7.5452841946279 1.42992959934712e-19
7.59525296412875 1.55468404823549e-19
7.6452217336296 1.68735721318623e-19
7.69519050313044 1.82215615235665e-19
7.74515927263129 1.95181280702952e-19
7.79512804213214 2.0679782643914e-19
7.84509681163298 2.16184402761377e-19
7.89506558113383 2.22495799281826e-19
7.94503435063468 2.25016625275889e-19
7.99500312013552 2.2325693678624e-19
8.04497188963637 2.1703457765373e-19
8.09494065913722 2.06528206375304e-19
8.14490942863807 1.92287344227377e-19
8.19487819813891 1.75192159291149e-19
8.24484696763976 1.56365071848338e-19
8.29481573714061 1.37046378463578e-19
8.34478450664145 1.18454142203802e-19
8.3947532761423 1.01652184075611e-19
8.44472204564315 8.74479995959819e-20
8.49469081514399 7.63353677732771e-20
8.54465958464484 6.84863752401659e-20
8.59462835414569 6.3787386854522e-20
8.64459712364654 6.19058141391073e-20
8.69456589314738 6.23710074243218e-20
8.74453466264823 6.46534573006672e-20
8.79450343214908 6.82306959215727e-20
8.84447220164992 7.26340799757184e-20
8.89444097115077 7.74762222153694e-20
8.94440974065162 8.24629228002755e-20
8.99437851015246 8.73954538593689e-20
9.04434727965331 9.21690137946688e-20
9.09431604915416 9.6771517256823e-20
9.14428481865501 1.01284209047777e-19
9.19425358815585 1.05882533583502e-19
9.2442223576567 1.10832964842417e-19
9.29419112715755 1.16479867516376e-19
9.34415989665839 1.23216620660775e-19
9.39412866615924 1.31437596316174e-19
9.44409743566009 1.41471991784215e-19
9.49406620516093 1.53506109209562e-19
9.54403497466178 1.67506014003619e-19
9.59400374416263 1.83155927267099e-19
9.64397251366348 1.99827878678586e-19
9.69394128316432 2.1659452889262e-19
9.74391005266517 2.3229016932209e-19
9.79387882216602 2.45616165143528e-19
9.84384759166686 2.55278495310084e-19
9.89381636116771 2.60138497594678e-19
9.94378513066856 2.59354833994416e-19
9.9937539001694 2.5249558422256e-19
10.0437226696703 2.39603899717456e-19
10.0936914391711 2.21207773911229e-19
10.1436602086719 1.98272817523429e-19
10.1936289781728 1.72105050250844e-19
10.2435977476736 1.44217419552865e-19
10.2935665171745 1.16178158351135e-19
10.3435352866753 8.9460708605462e-20
10.3935040561762 6.53136760042973e-20
10.443472825677 4.46654434512981e-20
10.4934415951779 2.80723252351059e-20
10.5434103646787 1.57124519546698e-20
10.5933791341796 7.42105536082761e-21
10.6433479036804 2.75754312912603e-21
10.6933166731813 1.09152222097774e-21
10.7432854426821 1.69411510782546e-21
10.793254212183 3.8223815980945e-21
10.8432229816838 6.78783718177028e-21
10.8931917511847 1.00041087086865e-20
10.9431605206855 1.30128120314968e-20
10.9931292901863 1.54900597411227e-20
11.0430980596872 1.72381844162712e-20
11.093066829188 1.81681807554837e-20
11.1430355986889 1.82781687368761e-20
11.1930043681897 1.76321514283394e-20
11.2429731376906 1.63418301854923e-20
11.2929419071914 1.4552578726665e-20
11.3429106766923 1.24331568982579e-20
11.3928794461931 1.01676090675746e-20
11.442848215694 7.94723219634821e-21
11.4928169851948 5.96062977055535e-21
11.5427857546957 4.38068951204787e-21
11.5927545241965 3.34868230210883e-21
11.6427232936974 2.95726361155749e-21
11.6926920631982 3.23553589039836e-21
11.7426608326991 4.14004942742305e-21
11.7926296021999 5.5553396568909e-21
11.8425983717007 7.30621571932898e-21
11.8925671412016 9.18172982145493e-21
11.9425359107024 1.0968061013848e-20
11.9925046802033 1.24851104937877e-20
12.0424734497041 1.36200982827579e-20
12.092442219205 1.43513494147972e-20
12.1424109887058 1.47568981596323e-20
12.1923797582067 1.50052668465766e-20
12.2423485277075 1.5329206579114e-20
12.2923172972084 1.59865576154455e-20
12.3422860667092 1.72149532397386e-20
12.3922548362101 1.91883093929953e-20
12.4422236057109 2.19826900461565e-20
12.4921923752118 2.5557342835332e-20
12.5421611447126 2.97538840337722e-20
12.5921299142134 3.43133793057373e-20
12.6420986837143 3.89080574345261e-20
12.6920674532151 4.31821533265636e-20
12.742036222716 4.67952545340435e-20
12.7920049922168 4.94616283613089e-20
12.8419737617177 5.09801989476928e-20
12.8919425312185 5.12517990431997e-20
12.9419113007194 5.02826059575026e-20
12.9918800702202 4.81748430424312e-20
13.0418488397211 4.51075261191502e-20
13.0918176092219 4.13110410107326e-20
13.1417863787228 3.70396016948967e-20
13.1917551482236 3.25452529469447e-20
13.2417239177245 2.8056238646392e-20
13.2916926872253 2.37614856413848e-20
13.3416614567262 1.98018636889944e-20
13.391630226227 1.62679320534169e-20
13.4415989957279 1.32031659763111e-20
13.4915677652287 1.06112075785752e-20
13.5415365347295 8.46550161993139e-21
13.5915053042304 6.71972848455099e-21
13.6414740737312 5.31769161095788e-21
13.6914428432321 4.20169897760205e-21
13.7414116127329 3.31893017579127e-21
13.7913803822338 2.62572365985302e-21
13.8413491517346 2.09007325218446e-21
13.8913179212355 1.69282369679489e-21
13.9412866907363 1.42806943998112e-21
13.9912554602372 1.30310213860243e-21
14.041224229738 1.33798113217568e-21
14.0911929992389 1.56451138108534e-21
14.1411617687397 2.02420699987096e-21
14.1911305382406 2.7647747274298e-21
14.2410993077414 3.83480289480752e-21
14.2910680772423 5.27666396514476e-21
14.3410368467431 7.11806139368586e-21
14.3910056162439 9.36307851225915e-21
14.4409743857448 1.19839248822219e-20
14.4909431552456 1.49147538773425e-20
14.5409119247465 1.80489053452194e-20
14.5908806942473 2.12406975427492e-20
14.6408494637482 2.4312459863282e-20
14.690818233249 2.70668826186966e-20
14.7407870027499 2.93040024149354e-20
14.7907557722507 3.08413153693391e-20
14.8407245417516 3.15347353266127e-20
14.8906933112524 3.12975524645575e-20
14.9406620807533 3.01143756805352e-20
14.9906308502541 2.80474001013942e-20
15.040599619755 2.52332745523439e-20
15.0905683892558 2.18702605510971e-20
15.1405371587566 1.81970375028157e-20
15.1905059282575 1.44660773265538e-20
15.2404746977583 1.09156218688671e-20
15.2904434672592 7.74466057799406e-21
15.34041223676 5.09479893929151e-21
15.3903810062609 3.04161202333736e-21
15.4403497757617 1.5962679657333e-21
15.4903185452626 7.16284666916241e-22
15.5402873147634 3.22672881525312e-22
15.5902560842643 3.19757366880111e-22
15.6402248537651 6.13824244322793e-22
15.690193623266 1.1273858150192e-21
15.7401623927668 1.80707404042221e-21
15.7901311622677 2.62465026368449e-21
15.8400999317685 3.57201033534254e-21
15.8900687012694 4.65209441133469e-21
15.9400374707702 5.86812411782149e-21
15.990006240271 7.21356950129514e-21
16.0399750097719 8.66479314275115e-21
16.0899437792727 1.0177595951306e-20
16.1399125487736 1.16880786716337e-20
16.1898813182744 1.31174846709471e-20
16.2398500877753 1.43800997801744e-20
16.2898188572761 1.53928975800619e-20
16.339787626777 1.60854395777403e-20
16.3897563962778 1.64085568478024e-20
16.4397251657787 1.63405355924544e-20
16.4896939352795 1.58898841016796e-20
16.5396627047804 1.50942441562551e-20
16.5896314742812 1.40155784016562e-20
16.6396002437821 1.27323443483354e-20
16.6895690132829 1.13298627676315e-20
16.7395377827838 9.8904058881646e-21
16.7895065522846 8.4845900751028e-21
16.8394753217854 7.16542258236857e-21
16.8894440912863 5.96584651176673e-21
16.9394128607871 4.89993979632476e-21
16.989381630288 3.96719065873283e-21
17.0393503997888 3.15865342920055e-21
17.0893191692897 2.4634294051436e-21
17.1392879387905 1.87390965044126e-21
17.1892567082914 1.38857274225593e-21
17.2392254777922 1.01177476238686e-21
17.2891942472931 7.507511829212e-22
17.3391630167939 6.10780507710435e-22
17.3891317862948 5.89954283031837e-22
17.4391005557956 6.7512857444972e-22
17.4890693252965 8.40359168102419e-22
17.5390380947973 1.0485098852761e-21
17.5890068642982 1.25592175696858e-21
17.638975633799 1.41924145082068e-21
17.6889444032998 1.50292923047508e-21
17.7389131728007 1.48574517496123e-21
17.7888819423015 1.36470204549437e-21
17.8388507118024 1.15552550327236e-21
17.8888194813032 8.89440451541867e-22
17.9387882508041 6.06918223633961e-22
17.9887570203049 3.49683132372836e-22
18.0387257898058 1.52642836226569e-22
18.0886945593066 3.74056231682685e-23
18.1386633288075 8.69603735894367e-24
18.1886320983083 5.4369643829723e-23
18.2386008678092 1.48995597338513e-22
18.28856963731 2.6027070147535e-22
18.3385384068109 3.56979246765982e-22
18.3885071763117 4.16906645842541e-22
18.4384759458126 4.33090994695694e-22
18.4884447153134 4.17049840564586e-22
18.5384134848142 3.98109217540642e-22
18.5883822543151 4.18622579377712e-22
18.6383510238159 5.25613490937228e-22
18.6883197933168 7.601061242937e-22
18.7382885628176 1.1460063085589e-21
18.7882573323185 1.68074235082358e-21
18.8382261018193 2.32986111071961e-21
18.8881948713202 3.02733336685646e-21
18.938163640821 3.68245745179544e-21
18.9881324103219 4.1930604092072e-21
19.0381011798227 4.46337789444196e-21
19.0880699493236 4.42381442523021e-21
19.1380387188244 4.04902546885564e-21
19.1880074883253 3.37064491641864e-21
19.2379762578261 2.48160220386694e-21
19.287945027327 1.53026368133246e-21
19.3379137968278 7.04364097301338e-22
19.3878825663286 2.06530108945796e-22
19.4378513358295 2.2477829273713e-22
19.4878201053303 9.02398021112669e-22
19.5377888748312 2.31193349205094e-21
19.587757644332 4.43753439801793e-21
19.6377264138329 7.16885134168576e-21
19.6876951833337 1.03080878983084e-20
19.7376639528346 1.35899935847426e-20
19.7876327223354 1.67127000684974e-20
19.8376014918363 1.93755759189928e-20
19.8875702613371 2.13189251301606e-20
19.937539030838 2.23596206417581e-20
19.9875078003388 2.24168758391645e-20
20.0374765698397 2.15234706717799e-20
20.0874453393405 1.98198612298518e-20
20.1374141088414 1.75314771057143e-20
20.1873828783422 1.49326656507109e-20
20.237351647843 1.23035113042175e-20
20.2873204173439 9.88744729128919e-21
20.3372891868447 7.85766250200354e-21
20.3872579563456 6.29861947351554e-21
20.4372267258464 5.20582570870927e-21
20.4871954953473 4.50304695540456e-21
20.5371642648481 4.07237470711237e-21
20.587133034349 3.7899045645444e-21
20.6371018038498 3.55890227965477e-21
20.6870705733507 3.33340289403632e-21
20.7370393428515 3.12784919989132e-21
20.7870081123524 3.01186178381903e-21
20.8369768818532 3.09260753251357e-21
20.8869456513541 3.48967141479769e-21
20.9369144208549 4.30835001692164e-21
20.9868831903557 5.61682633154029e-21
21.0368519598566 7.43109988482424e-21
21.0868207293574 9.70941581594066e-21
21.1367894988583 1.2355871169905e-20
21.1867582683591 1.52313368282084e-20
21.23672703786 1.81690326022313e-20
21.2866958073608 2.09919974739334e-20
21.3366645768617 2.35301023210891e-20
21.3866333463625 2.56348926067388e-20
21.4366021158634 2.71911936444968e-20
21.4865708853642 2.81249250064536e-20
21.5365396548651 2.84069202712933e-20
21.5865084243659 2.80527761472628e-20
21.6364771938668 2.71189354351368e-20
21.6864459633676 2.5695406525365e-20
21.7364147328685 2.38957678826841e-20
21.7863835023693 2.1845385393293e-20
21.8363522718701 1.96690332940263e-20
21.886321041371 1.74792860977447e-20
21.9362898108718 1.53670716673567e-20
21.9862585803727 1.33955986241451e-20
22.0362273498735 1.1598486314742e-20
22.0861961193744 9.98236903610483e-21
22.1361648888752 8.53359627492429e-21
22.1861336583761 7.22801386981547e-21
22.2361024278769 6.0423008918156e-21
22.2860711973778 4.96505011115184e-21
22.3360399668786 4.00577405433882e-21
22.3860087363795 3.20030019032753e-21
22.4359775058803 2.61154124896145e-21
22.4859462753812 2.32530146174398e-21
22.535915044882 2.44149553111955e-21
22.5858838143829 3.06180426115325e-21
22.6358525838837 4.27528106501893e-21
22.6858213533845 6.14370480701254e-21
22.7357901228854 8.68853309496124e-21
22.7857588923862 1.18811637337009e-20
22.8357276618871 1.56378948469108e-20
22.8856964313879 1.98205224823764e-20
22.9356652008888 2.42429583309838e-20
22.9856339703896 2.86836153525976e-20
23.0356027398905 3.29026291873715e-20
23.0855715093913 3.66623147605209e-20
23.1355402788922 3.97486886461358e-20
23.185509048393 4.19915316958061e-20
23.2354778178939 4.32804361254618e-20
23.2854465873947 4.35746515247918e-20
23.3354153568956 4.29053145114956e-20
23.3853841263964 4.1369720486486e-20
23.4353528958973 3.91184889262985e-20
23.4853216653981 3.63375525376295e-20
23.5352904348989 3.32276418320752e-20
23.5852592043998 2.99841869614491e-20
23.6352279739006 2.67802770217803e-20
23.6851967434015 2.37545847358298e-20
23.7351655129023 2.10051601248376e-20
23.7851342824032 1.85889462287867e-20
23.835103051904 1.65259868851973e-20
23.8850718214049 1.48067328901222e-20
23.9350405909057 1.34006708991734e-20
23.9850093604066 1.22646713116297e-20
24.0349781299074 1.13498811000557e-20
24.0849468994083 1.06065456948189e-20
24.1349156689091 9.98670322749734e-21
24.18488443841 9.44515553622952e-21
24.2348532079108 8.93942440195218e-21
24.2848219774117 8.42952805448902e-21
24.3347907469125 7.87836953716225e-21
24.3847595164133 7.25333797934052e-21
24.4347282859142 6.52941608266886e-21
24.484697055415 5.69369735525629e-21
24.5346658249159 4.75078935401773e-21
24.5846345944167 3.72817366737918e-21
24.6346033639176 2.68028463943243e-21
24.6845721334184 1.68994016828027e-21
24.7345409029193 8.65883614114975e-22
24.7845096724201 3.35617848103373e-22
24.834478441921 2.33417477733586e-22
24.8844472114218 6.84310920024607e-22
24.9344159809227 1.7857807712362e-21
24.9843847504235 3.58974322114064e-21
25.0343535199244 6.0878311613303e-21
25.0843222894252 9.202956285715e-21
25.1342910589261 1.27894830715752e-20
25.1842598284269 1.66431499739743e-20
25.2342285979277 2.05202854780084e-20
25.2841973674286 2.41641616562598e-20
25.3341661369294 2.73348392406e-20
25.3841349064303 2.98379135203974e-20
25.4341036759311 3.15474176732735e-20
25.484072445432 3.24188844675956e-20
25.5340412149328 3.24901332174072e-20
25.5840099844337 3.18694805312026e-20
25.6339787539345 3.07133755809632e-20
25.6839475234354 2.91974751176282e-20
25.7339162929362 2.74864753801668e-20
25.7838850624371 2.57083109886744e-20
25.8338538319379 2.39375185440247e-20
25.8838226014388 2.21907786295012e-20
25.9337913709396 2.0435237205381e-20
25.9837601404405 1.860765000404e-20
26.0337289099413 1.66402149017131e-20
26.0836976794421 1.44876079828714e-20
26.133666448943 1.21494994083597e-20
26.1836352184438 9.68374454414321e-21
26.2336039879447 7.20733212686346e-21
26.2835727574455 4.8846333840661e-21
26.3335415269464 2.90502822216088e-21
26.3835102964472 1.45407430934355e-21
26.4334790659481 6.83620087365274e-22
26.4834478354489 6.86415472369635e-22
26.5334166049498 1.47984282208713e-21
26.5833853744506 3.00159424922779e-21
26.6333541439515 5.11784547827757e-21
26.6833229134523 7.64218830002435e-21
26.7332916829532 1.03617315480639e-20
26.783260452454 1.306570186377e-20
26.8332292219549 1.5571732649261e-20
26.8831979914557 1.77457858808865e-20
26.9331667609565 1.95130913573073e-20
26.9831355304574 2.08592746217422e-20
27.0331042999582 2.18226001362733e-20
27.0830730694591 2.24796444331265e-20
27.1330418389599 2.29275089414731e-20
27.1830106084608 2.32658053778341e-20
27.2329793779616 2.35811678011704e-20
27.2829481474625 2.39361502776534e-20
27.3329169169633 2.43633035177677e-20
27.3828856864642 2.48642311791919e-20
27.432854455965 2.54126929777581e-20
27.4828232254659 2.59604424835072e-20
27.5327919949667 2.64444591517118e-20
27.5827607644676 2.67944723319142e-20
27.6327295339684 2.69400490145866e-20
27.6826983034693 2.68168909661061e-20
27.7326670729701 2.63722555290253e-20
27.7826358424709 2.55695259998133e-20
27.8326046119718 2.43919191264359e-20
27.8825733814726 2.28451836821319e-20
27.9325421509735 2.09589958031154e-20
27.9825109204743 1.87866740401322e-20
28.0324796899752 1.64028767509919e-20
28.082448459476 1.38991254363959e-20
28.1324172289769 1.1377296928926e-20
28.1823859984777 8.94158675662002e-21
28.2323547679786 6.68978643639975e-21
28.2823235374794 4.70495730334598e-21
28.3322923069803 3.04865680975611e-21
28.3822610764811 1.75674431669242e-21
28.432229845982 8.38465265952326e-22
28.4821986154828 2.79028230825107e-22
28.5321673849836 4.53254987207942e-23
28.5821361544845 9.39026207774983e-23
28.6321049239853 3.79858041758673e-22
28.6820736934862 8.65137931121401e-22
28.732042462987 1.52473851680087e-21
28.7820112324879 2.34964774216169e-21
28.8319800019887 3.34590797411435e-21
28.8819487714896 4.52987287415785e-21
28.9319175409904 5.92044247110332e-21
28.9818863104913 7.52965786255854e-21
29.0318550799921 9.35340250642347e-21
29.081823849493 1.13640119245189e-20
29.1317926189938 1.35063165399043e-20
29.1817613884947 1.56980764039094e-20
29.2317301579955 1.78350127158017e-20
29.2816989274964 1.97998390332055e-20
29.3316676969972 2.14739954676952e-20
29.381636466498 2.27503225868885e-20
29.4316052359989 2.35447621910993e-20
29.4815740054997 2.38053603238673e-20
29.5315427750006 2.35173284347244e-20
29.5815115445014 2.27035914107389e-20
29.6314803140023 2.14209840694497e-20
29.6814490835031 1.97529149461922e-20
29.731417853004 1.77997861960578e-20
29.7813866225048 1.56686742793611e-20
29.8313553920057 1.34637250932343e-20
29.8813241615065 1.12784378797318e-20
29.9312929310074 9.19058080617042e-21
29.9812617005082 7.25999152927659e-21
};
\addlegendentry{$\;I(\w) = I_{\parallel}(\w) + I_{\bot}(\w)$}
\end{axis}

\end{tikzpicture}

\\[1.5em]
\input{Emission_para_ortho.tikz}
\caption{Emission intensity from the irradiated material computed from Eqs.~\eqref{emission} and~\eqref{emissiondecomp2}.
%
The frequency is given by $f\eqt\omega/(2\pi)$.
}
    \label{fig:emission}
\end{figure}
Experiments measure the  frequency resolved emission intensity~$I$, which is computed by Eq.~(53) in Ref.~[\ref{Wilhelm2021}]
\begin{align}
I(\omega) &=
\frac{\omega^2}{3c^3}\,|\bj(\omega)|^2\;, \label{emission}
\end{align}
where $\bj(\omega)$ is the Fourier transform of $\bj(t)$ and $c$ is the speed of light. 
%
The emission is sketched in Fig.~\ref{fig:emission}.
%
When inserting Eq.~\eqref{currentdecomp} in the frequency domain, we have
\begin{align}
I(\omega) &=
\frac{\omega^2}{3c^3}\,\Big(\,|\,\bj_\parallel(\omega)\sd|^2
+ |\,\bj_\bot(\omega)\sd|^2 \,\Big)
\;, \label{emissiondecomp}
\end{align}
that motivates the definitions
\begin{align}
I_\parallel(\omega) =
\frac{\omega^2}{3c^3}\, |\sd\bj_\parallel(\omega)\sd|^2\;,
\hspace{2em}
I_\bot(\omega) =
\frac{\omega^2}{3c^3}\, |\sd\bj_\bot(\omega)\sd|^2\;. \label{emissiondecomp2}
\end{align}
We recover $I(\omega)\eqt I_\parallel(\omega)\,{+}\sd I_\bot(\omega)$.




\section{References}
When using the CUED software package, please reference to CUED by citing the following publication:
\begin{enumerate}[leftmargin=*]

\paper{J.~Wilhelm, P.~Grössing, A.~Seith, J.~Crewse, M.~Nitsch, L.~Weigl, C.~Schmid, and F.~Evers}{Semi\-con\-duc\-tor-Bloch Formalism: Derivation and Application to High-Harmonic Generation from Dirac Fermions}{TOBEFILLED}{ 
Phys.~Rev.~B~\,\textbf{x}, y (2021)}
\label{Wilhelm2021}

\end{enumerate}

\end{document}
