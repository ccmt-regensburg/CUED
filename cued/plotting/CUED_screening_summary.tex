\documentclass[11pt, a4paper]{scrartcl}
\usepackage{xcolor}
\usepackage[colorlinks=true,allcolors=blue,pdfborder={0 0 0},pdfstartview={FitV},breaklinks,linktocpage]{hyperref}
\usepackage{color,amsmath,amsfonts,amssymb}
\usepackage[utf8]{inputenc}
\usepackage{enumitem}%für mehrspaltige Aufzählungen
\usepackage[ngerman, english]{babel}
\usepackage{epsfig}
\usepackage{pstricks}
\usepackage{graphics}
\graphicspath{{Figures/}}
\usepackage{bbm}
%\usepackage{showlabels}
\usepackage[labelfont=bf]{caption}
%\usepackage{nonfloat}
\usepackage{siunitx}
\DeclareSIUnit{\atomicunits}{a.u.}
\usepackage{booktabs}%für Tabellen
\usepackage{pgffor} %for loops
\usepackage{ifthen} %for if
\usepackage{pgfplots} %für Plots
\pgfplotsset{
compat=1.5,
tick label style={font=\small},
every axis/.append style={line width=0.8pt, tick style={line width=0.7pt}}
} %für richtigen Abstand der labels
\usetikzlibrary{shapes,arrows,fit,calc,positioning,arrows}
\usepackage[
  textheight=25cm,
left=2.3cm,
right=2.3cm,
headheight=25pt,
  includehead,includefoot,
  heightrounded,
]{geometry}
\makeatletter
\def\input@path{{./Figures/}}
\usepackage{fancyhdr}
\pagestyle{fancy}
\fancyfoot{}
\fancyhead{}
% \renewcommand{\chaptermark}[1]
% {\markboth{{\chaptername\ \thechapter:\  #1}}{}}
% \renewcommand{\sectionmark}[1]
% {{\markright{ \thesection\ #1}}} 
%\renewcommand{\sectionmark}[1]{\markboth{#1}{}}

%\renewcommand\sectionmark[1]{\markright{#1}}
%\renewcommand{\thesection}{\arabic{section}.}
\fancyhead[L]{}
\fancyhead[R]{\thepage}
\fancyhead[C]{\nouppercase{\leftmark}}
%\fancyhead[C]{\nouppercase{\rightmark}}


%\usepackage{sectsty}
\usepackage{url}
%\usepackage{breakurl}
\usepackage{amsthm}
\usepackage{graphicx}
\usepackage{tikzpagenodes}

\definecolor{darkgreen}{rgb}{0.0,0.6,0.0}
\definecolor{shadedgreen}{rgb}{0.8,0.95,0.8}
\definecolor{shadedblue}{rgb}{0.85,0.85,1.0}
\newlength\figureheight 
\newlength\figurewidth


\usepackage[varg]{txfonts} %font/Schrift
%\usepackage{csquotes}
% \usepackage[minnames=3,maxnames=30,backend=biber,natbib=true,sorting=none,citestyle=numeric-comp,giveninits=true]{biblatex}
% \addbibresource{Literature.bib}
% %\setlength\bibitemsep{0.5em}
% \setlength{\biblabelsep}{0.4em}

% \DeclareFieldFormat{pages}{#1}
% \DeclareFieldFormat{journaltitle}{#1}
% \DeclareFieldFormat{eprint}{#1}
% \DeclareFieldFormat[article]{title}{\it#1\isdot}

% \DeclareBibliographyDriver{article}{
% \printnames{author}, \printfield{title},
% \href{http://dx.doi.org/\thefield{doi}}
% {\printfield{journaltitle} 
% \textbf{\printfield{volume}}, \printfield{pages} 
% (\printfield{year})}.}

% \DeclareBibliographyDriver{misc}{
% \printnames{author}, \printfield{title},
% \href{https://arxiv.org/abs/\thefield{eprint}}{
% arXiv preprint, arXiv:\printfield{eprint}  (\printfield{year})}.}

% \DeclareBibliographyDriver{inbook}{\hspace{-0.1cm}
% \printnames{author},
% \printfield{title},
% in \printfield{booktitle}.
% \printlist{publisher}
% (\printfield{year}).
% }


% \renewbibmacro*{name:andothers}{% Based on name:andothers from biblatex.def
%   \ifboolexpr{
%     test {\ifnumequal{\value{listcount}}{\value{liststop}}}
%     and
%     test \ifmorenames
%   }
%     {\ifnumgreater{\value{liststop}}{1}
%       {\finalandcomma}
%       {}%
%      \andothersdelim\bibstring[\emph]{andothers}}
%     {}}

\setenumerate[1]{label=[\arabic*],ref=\arabic*}
\newcommand{\paper}[4]{\item #1, \,\textit{#2}, \,\href{#3}{#4}.\\[-1.4em]}


\newcommand{\bE}{\mathbf{E}}
\newcommand{\bk}{\mathbf{k}}
\newcommand{\bA}{\mathbf{A}}
\newcommand{\bj}{\mathbf{j}}
\newcommand{\bd}{\mathbf{d}}
\newcommand{\sd}{\hspace{0.05em}}
\newcommand{\eqt}{\,{=}\,}
\newcommand{\pt}{\,{+}\,}
\newcommand{\coloneqqt}{\,{\coloneqq}\,}
\newcommand{\intbzdkpi}{\int\limits_\text{BZ}\frac{d\bk}{(2\pi)^2}}
\newcommand{\rhonnprime}{\rho_{nn'}(\bk,t)}
\newcommand{\rhonprimen}{\rho_{n'n}(\bk,t)}
\newcommand{\un}{\underline{n}}
\newcommand{\braunk}{\langle u_{n\bk}|}
\newcommand{\ketunk}{|u_{n\bk}\rangle}
\newcommand{\ketunkprime}{|u_{n'\bk}\rangle}
\newcommand{\timest}{\,{\times}\,}

\definecolor{darkgreen}{rgb}{0.0,0.5,0.0}


\begin{document}

\pagenumbering{Roman}

\begin{titlepage}\pdfbookmark[0]{Title}{Title}
  \sffamily
  \begin{center}
{
\includegraphics[width=9cm]{logo.pdf}
\\[5em]
\Huge \bfseries Screening summary of the CUED program}
\\[3em]\large
The CUED program is developed and maintained by:
\\[3em]
Chair of Computational Condensed Matter Theory
  \\[0.5em]
Institute of Theoretical Physics
  \\[0.5em]
University of Regensburg
  \\[0.5em]
Universitätsstraße 31
  \\[0.5em]
D\,-\,93053 Regensburg
  \\[0.5em]
Germany
\\[3em]
Date of execution: \today
  \\[0.5em]
  \end{center}{\large
  Contact:
  \\[1em]
    Jan Wilhelm
      \\[0.5em]
    Ferdinand Evers
  \\[3em]
  Contributors (in alphabetic order): 
  \\[1em]
  Jack Crewse
  \\[0.5em]
  Patrick Grössing
  \\[0.5em]
  Adrian Seith
  }
\end{titlepage}

\pagenumbering{arabic}
\pagestyle{plain}

\pdfbookmark[0]{Contents}{Contents}
\tableofcontents

\pagestyle{fancy}

\section{Screening parallel \& orthogonal Emission}
\begin{figure}
    \centering
    \includegraphics[width=\textwidth]{PH-EDIR-PLOT}
    \caption{Screening plot of PH-PARAMETER against frequency. The maximum intensity in electric
      field direction is $I_{\mathrm{hh}}^{\mathrm{max}} = \SI{PH-EDIR-IMAX}{[\atomicunits]}$.}
    \label{fig:sec1_parallel_screening}
\end{figure}
\begin{figure}
    \centering
    \includegraphics[width=\textwidth]{PH-ORTHO-PLOT}
    \caption{Screening plot of PH-PARAMETER against frequency. The maximum intensity orthogonal
      to the electric field direction is $I_{\mathrm{hh}}^{\mathrm{max}} = \SI{PH-ORTHO-IMAX}{[\atomicunits]}$.}
    \label{fig:sec1_orthogonal_screening}
\end{figure}


\section{References}
When using the CUED software package, please reference to CUED by citing the following publication:
\begin{enumerate}[leftmargin=*]

\paper{J.~Wilhelm, P.~Grössing, A.~Seith, J.~Crewse, M.~Nitsch, L.~Weigl, C.~Schmid, and F.~Evers}{Semi\-con\-duc\-tor-Bloch Formalism: Derivation and Application to High-Harmonic Generation from Dirac Fermions}{https://doi.org/10.1103/PhysRevB.103.125419}{ 
Phys.~Rev.~B~\,\textbf{103}, 125419 (2021)}
\label{Wilhelm2021}

\end{enumerate}

\end{document}